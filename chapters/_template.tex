\section{
  % Überschrift/Name des Rezepts
  Tolles Rezept
}
% Der Text im 'label'-Feld erlaubt es im Dokument auf dieses Rezept zu verweisen.
% Sehr praktisch, wenn z.B. ein anderes Rezept dieses hier verwendet. Dabei werden Seitenzahl etc. automatisch
% angepasst.
\label{rcp:tolles-rezept}

% Das Folgende ist ein Einführungstext mit etwas Zusatzinfos:
\begin{recipeintro}{
% Anzahl der Portionen
4
}{
% Zubereitsungsdauer
1
}[
% Optional: Quellenangabe
https://www.example.com
]
% Einführungstext
Super geiles Rezept!
\end{recipeintro}

% Ein Zutatenblock mit Überschrift
% Bei längeren/komplizierteren Rezepten bietet es sich an, mehrere solcher Blöcke zu verwenden
% um die Zutanten semantisch nach Rezeptmodule aufzuteilen.
\begin{ingredients}{
  % Überschrift
  Zutaten
  }
% Zutanten sind in einer Tabelle aufgelistet. Die Spalten werden durch ein '&' Zeichen getrennt,
% Eine Zeile wird mit '\\' abgeschlossen.

% <Menge>     &     <Einheit>          &  <Name>     \\
2                &  \si{\tl}           &  Szechuanpfeffer, ganz  \\

% Der Befehl '\si{<einheit>}' sollte wenn möglich für Einheiten verwendet werden.
% Wenn Einheiten fehlen, kann ich sie hinzufügen. FÜr mehr Infos: Dokumentation Paket siunitx.
% Beispiele:

% Teelöffel:   \si{\tl}
% Esslöffel:   \si{\el}
% Milliliter:  \si{\milli\litre}
% Kilogram:    \si{\kilo\gram}
\end{ingredients}

% Ein bissl Luft zwischen Rezeptliste und Anweisungen.
\vspace{0.5cm}

% Ein Zubereitungsschritt mit Überschrift
% Diese Blöcke können beliebig oft nach einander erscheinen
\begin{recipestep}{
  % Überschrift
  Schritt 1
  }
  % Der Text, der die Zubereitung beschreibt:
  Bla bla bla...
\end{recipestep}
