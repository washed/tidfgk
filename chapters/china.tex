\chapter{China}
Dolle Rezepte aus China!

\section{Szechuan Dan Dan Nudeln}

\begin{recipeintro}{2}{\nicefrac{1}{2}}[https://www.marionskitchen.com]
  Turbo scharf!
\end{recipeintro}

\begin{ingredients}{Scharfe Sauce}
2                &  \si{\tl}            &  Szechuanpfeffer, ganz  \\
60               &  \si{\milli\litre}   &  Sojasauce              \\
3                &  \si{\el}            &  Chilliöl               \\
1                &  \si{\el}            &  Schwarzer chinesischer Essig  \\
2                &  \si{\tl}            &  Süße dunkle Sojasauce  \\
\nicefrac{1}{4}  &  \si{\tl}            &  Zucker  \\
\end{ingredients}

\begin{ingredients}{Stir-Fry}
2      &  \si{\el}            &  Pflanzenöl (z.B. Erdnussöl)  \\
200    &  \si{\gram}          &  Schweinehack \\
2      &  \si{\el}            &  Tianjin eingelegte Senfblätter (Ya Chai) oder Kohl (Tong Chai) \\
1      &  \si{\el}            &  Sojasauce  \\
1      &  \si{\el}            &  Shaoxing (Chinesischer Kochwein)  \\
\end{ingredients}

\begin{ingredients}{Anrichten}
200    &  \si{\gram}         &  Dünne, chinesische Weizennudeln  \\
       &                 &  Frühlingszwiebeln zum Anrichten  \\
\end{ingredients}

\vspace{0.5cm}

\begin{recipestep}{Scharfe Sauce}
Den Szechuanpfeffer in einer Pfanne \SIrange{2}{3}{\minute} ohne Öl bei mittlerer Hitze rösten bis er beginnt zu duften.
In einem Mörser fein zerstoßen und beiseite stellen. Alle anderen Zutaten in einer großen Schüssel verühren und beiseite stellen.
\end{recipestep}

\begin{recipestep}{Stir-Fry}
Das Öl in einem Wok oder einer großen Pfanne auf hoher Hitze stark erhitzen. Das Hackfleisch hineingeben und verteilen.
Bei größeren Mengen unbedingt nicht zu viel Fleisch auf einmal verwenden! Es muss stark angebraten werden und nicht dünsten!
Unter gelegentlichem Rühren kräftig anbraten, bis es eine schöne gold-braune Farbe hat. Nun stärker rühren und das Fleisch fein
auftrennen, schließlich den gerösteten und zerstoßenen Szechuanpfeffer einarbeiten. Die restlichen Zutaten dazugeben und unter
ständigem Rühren weiterbraten so dass das Fleisch die Flüssigkeit aufnehmen kann. Beiseite stellen und warm halten, während die Nudeln kochen.

Einen großen Topf Wasser zum Kochen bringen und die Nudeln nach Packungsanweisungen kochen. Während die Nudeln kochen, einen kleinen Schöpfer
Nudelwasser (etwa \SIrange{50}{100}{\milli\litre}) entnehmen und mit der Soße vermischen. Die Frühlingszwiebeln in feine Ringe schneiden.
\end{recipestep}

\begin{recipestep}{Anrichten}
Etwas scharfe Soße in tiefe Teller oder Schüsseln geben, die Nudeln darin zu einem Nest drehen.
Die Fleischmischung darauf anrichten und mit den Frühlingszwiebeln garnieren. Übrige scharfe Soße und Chilliöl
zum Nachwürzen anbieten.
\end{recipestep}

\section{Szechuan Chilliöl}\label{rcp:szechuan-chillioel}
\begin{recipeintro}{2}{\nicefrac{1}{4}}[https://www.marionskitchen.com]
  Universelles, chinesisches Chilliöl.
\end{recipeintro}

\begin{ingredients}{Zutaten}
  200  &  \si{\milli\litre}  &  Erdnussöl  \\
  4    &  Stk.               &  Zimtstange  \\
  2    &  Stk.               &  Lorbeerblätter  \\
  3    &  \si{\el}           &  Szechuan-Pfefferkörner  \\
  8    &  Stk.               &  Kardamomkapseln  \\
  80   &  \si{\milli\litre}  &  Asiatische Chilliflocken  \\
  2    &  \si{\tl}           &  Salz  \\
\end{ingredients}

\vspace{0.5cm}

\begin{recipestep}{Zubereitung}
In einer Pfanne das Erdnussöl bei geringer Hitze erwärmen.\par

Die Gewürze in einer weiteren Pfanne ohne Öl bei mittlerer Hitze rösten, gerade bis sie beginnen leicht zu rauchen.
In die Pfanne mit dem warmen Öl geben, unter mittlerer Hitze erhitzen und etwa \SIrange{4}{5}{\minute} sanft braten.
Beiseite stellen und abkühlen lassen.\par

Chilliflocken und Salz in eine hitzefeste Schüssel oder Glas geben (z.B. Einmachglas) und das Öl hineinseien.
Lorbeerblätter, Zimtstange und Sternanis ebenfalls hinzufügen.\par
\end{recipestep}

\section{Blanchierter Pak Choi}\label{rcp:pakchoi-blanchiert}
\begin{recipeintro}{2}{\nicefrac{1}{4}}
  Eine klassische chinesische Beilage.
\end{recipeintro}

\begin{ingredients}{Zutaten}
  1  &  Stk.      &  Pak Choi  \\
  1  &  \si{\el}  &  Austernsauce  \\
  1  &  \si{\tl}  &  Zucker  \\
  1  &  \si{\el}  &  Erdnussöl \\
  1  &  \si{\el}  &  Wasser \\
\end{ingredients}

\vspace{0.5cm}

\begin{recipestep}{Zubereitung}
Einen Topf mit reichlich gesalzenem Wasser zum Kochen bringen. In der Zwischenzeit die Austernsauce, den Zucker, das Wasser und das Erdnussöl
in einem Topf oder Wok unter Rühren erwärmen bis sich der Zucker vollständig aufgelöst hat.\par

Die Blätter des Pak Choi abziehen, waschen und je nach Größe vierteln oder halbieren. Eine Minute blanchieren, herausnehmen und dann sofort
mit der lauwarmen Sauce vermischen.
\end{recipestep}
