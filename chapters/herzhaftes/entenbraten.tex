  % Überschrift/Name des Rezepts
\section{Entenbraten}\label{rcp:entenbraten}

% Der Text im 'label'-Feld erlaubt es im Dokument auf dieses Rezept zu verweisen.
% Sehr praktisch, wenn z.B. ein anderes Rezept dieses hier verwendet. Dabei werden Seitenzahl etc. automatisch
% angepasst.

% Das Folgende ist ein Einführungstext mit etwas Zusatzinfos:
\begin{recipeintro}
  % Anzahl der Portionen (optional)
  [2-4 & Portionen]
  % Zubereitsungsdauer (optional)
  [1 & Stunde]
  % Optional: Quellenangabe
  [\url{https://www.example.com}]

  % Einführungstext
  Super geiles Rezept!
\end{recipeintro}

% Ein Zutatenblock mit Überschrift
% Bei längeren/komplizierteren Rezepten bietet es sich an, mehrere solcher Blöcke zu verwenden
% um die Zutanten semantisch nach Rezeptmodule aufzuteilen.
\begin{ingredients}
  {Braten}
  1 & & Ente  \\
  1 & & Apfel  \\
  1 & & Orange, unbehandelt  \\
  1 & & Petersilienwurzel  \\
  1 & \si{\tl} & Szechuan Pfeffer  \\
  2-3 & & Schalotten  \\
  100 & \si{\gram} & Majoran frisch/gefroren  \\
  & & Olivenöl \\
  & & Salz \\
  & & Pfeffer schwarz, frisch gemahlen  \\

  % Teelöffel:   \si{\tl}
  % Esslöffel:   \si{\el}
  % Milliliter:  \si{\milli\litre}
  % Kilogram:    \si{\kilo\gram}
\end{ingredients}

% Ein bissl Luft zwischen Rezeptliste und Anweisungen.
\vspace{0.5cm}

% Ein Zubereitungsschritt mit Überschrift
% Diese Blöcke können beliebig oft nach einander erscheinen
\begin{recipestep}
  % Überschrift
  {Zubereitung}
  % Der Text, der die Zubereitung beschreibt:
  Bla bla bla\ldots
\end{recipestep}
