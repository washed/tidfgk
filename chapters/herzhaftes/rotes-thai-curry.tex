\section{Rotes Thai-Curry - Kaeng Phet}\label{rcp:rotes-thai-curry}%
\index{Hauptgericht}\index[region]{Thailand}\index[ingredient]{Currypaste}\index[ingredient]{Kokosmilch}\index[ingredient]{Fischsauce}
\begin{recipeintro}
  [4 & Portionen]
  [\nicefrac{1}{2} & Stunde]
  []
  Einfach, schnell
\end{recipeintro}

\begin{ingredients}{Curry}
  50-100           &  \si{\gram}          &  Rote Currypaste  \\
  1                &  \si{\litre}         &  Kokosmilch              \\
  2                &  \si{\el}            &  Palmzucker  \\
  4                &  \si{\el}            &  Fischsauce  \\
  4                &  Stk.                &  Kaffirlimettenblätter  \\
  1                &  Dose/Glas           &  Bambussprossen in Scheiben  \\
  500              &  \si{\gram}          &  Hühnerbrust (oder anderes Fleisch, Fisch, Krustentiere, Tofu möglich)  \\
  2 - 4            &  Stk.                &  Rote Thai-Chilies, frisch (optional)  \\
                   &                      &  Gemüse nach Wahl, z.B. Zuckerschoten, Austernpilze, Champignons, Karotten, Paprika ...  \\
                   &                      &  Koriandergrün (optional, zum Servieren)  \\
                   &                      &  Salz  \\
\end{ingredients}

\begin{ingredients}{Reis}
  3 - 4            &  Tassen              & \nameref{rcp:jasmin-reis}
\end{ingredients}

\vspace{0.5cm}

\begin{recipestep}{Curry}
  Den Reis wie im Rezept\footnote{\fullref{rcp:jasmin-reis}} beschrieben aufsetzen.\par
  Eine kleine Menge Kokosmilch (50 - 100 \si{\milli\litre}) in einem Topf erhitzen, die Currypaste darin auflösen
  und unter gelegentlichem Rühren einige Minuten leicht köcheln lassen, bis sich das Öl etwas vom Rest trennt.\par

  Währenddessen das Gemüse putzen und in ca 5 x 1 \si{\centi\metre} große Streifen schneiden, Bambussprossen abseien.
  Fleisch gegebenfalls parieren und in mundgerechte Stücke schneiden.\par

  Die restliche Kokosmilch angießen, mit Fischsauce, Palmzucker und Salz abschmecken. Die Kaffirlimettenblätter und
  das Fleisch dazugeben (wenn gewünscht, 1-2 Kaffirlimettenblätter zum Anrichten zurückhalten).
  Das Curry sollte ab jetzt nicht mehr stark kochen, sondern lediglich leicht ziehen.
  Wer es etwas schärfer mag, kann jetzt noch frische Thai-Chilies, in Ringe geschnitten, dazugeben.\par

  Das Gemüse je nach Garzeit nach und nach dazugeben.

  Noch einmal mit Fischsauce, Palmzucker und Salz abschmecken.
\end{recipestep}

\begin{recipestep}{Anrichten}
  Entweder Reis und Curry in getrennten Schalen servieren, oder zusammen in einem tiefen Teller.
  Koriander waschen, Blätter abzupfen und trocken tupfen. Übrige Kaffirlimettenblätter in dünne Streifen schneiden,
  Chilies, falls gewünscht, in dünne Ringe schneiden und das Curry damit garnieren.
\end{recipestep}
