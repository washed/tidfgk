\section{Jasmin Reis}\label{rcp:jasmin-reis}%
\index{Beilage}\index[region]{China}\index[region]{Thailand}\index[region]{Vietnam}\index[ingredient]{Reis}
\begin{recipeintro}
  []
  [25 & Minuten]
  []
  Reis - it's complicated
\end{recipeintro}

\begin{ingredients}{Zutaten}
  1                                    &  Einheit      &  Jasmin-Reis  \\
  1\nicefrac{1}{6} - 1\nicefrac{1}{3}  &  Einheit      &  Wasser       \\
\end{ingredients}

\vspace{0.5cm}

\begin{recipestep}{Zubereitung}
  Dieses Rezept funktionert für etwa 1 bis 4 Tassen Reis.
  Wenn ihr einen Reiskocher besitzt, ignoriert dieses Rezept und benutzt den bitte!\par

  Den Reis in einen kleinen Topf geben, der mindestens genug Platz für die ca. 3 fache Menge Reis
  hätte, denn er quillt beim Kochen sehr auf. Den Reis mehrmals mit kalten Wasser waschen.
  Dabei immer etwas Wasser zum Reis geben, mit den Fingern etwas durchmischen und das Wasser
  vorsichtig abgießen. Das 2-4 Mal wiederholen, bis das Wasser weitestgehend klar ist.
  Nun den Reis in einem Sieb gut abtropfen lassen.\par

  Den abgetropften Reis wieder in den Topf geben und mit der entsprechenden Menge Wasser auffüllen.
  Die exakte Menge ist je nach Sorte, Alter und vielen weitere Faktoren immer etwas unterschiedlich,
  aber dieses Rezept gibt einen Bereich an, der in den meisten Fällen gut funktionieren sollte.
  Für beispielsweise 3 Tassen Reis verwenden wir \(3 \times 1\nicefrac{1}{6}\) bis \(3 \times 1\nicefrac{1}{3}\),
  also 3,5 bis 4 Tassen Wasser. Das Wasser zum Reis geben und wenn gewünscht mit etwas
  Salz (ca. \nicefrac{1}{4} \si{tl} pro Tasse) würzen.

  Den Topf auf den Herd stellen und auf hoher Hitze mit geschlossenem Deckel zum Kochen bringen.
  Wenn das Wasser kocht, die Hitze stark reduzieren, so dass es mit geschlossenem Deckel nur
  noch leicht köchelt. Bei konventionellen Elektroherden könnt ihr hier meist bereits jetzt
  schon komplett abschalten und den Topf auf der heißen Platte stehen lassen. Auf Induktionsherden
  muss man hier ein bisschen experimentieren. Nach 10 Minuten den Herd komplett abschalten und den Reis
  weitere 10 Minuten bei geschlossenem Deckel quellen lassen. Anschließend den Deckel öffnen und
  den Reis mit einer Gabel oder einem Reislöffel auflockern und den Dampf entweichen lassen.
\end{recipestep}
