\documentclass[10pt,hidelinks]{book}

\usepackage[a4paper,left=3cm,right=3cm,top=3cm,bottom=3cm,bindingoffset=10mm]{geometry}
\usepackage{nicefrac}

\usepackage[sfdefault,light]{roboto}  %% Option 'sfdefault' only if the base font of the document is to be sans serif
\usepackage[T1]{fontenc}

\usepackage{url}
\usepackage[svgnames]{xcolor}
\usepackage{siunitx}
\usepackage{xparse}

\usepackage{multirow}

% This fixes lseries error with siunitx and light fonts
\usepackage{xpatch}
\ExplSyntaxOn
\xpatchcmd{\__siunitx_detect_font_weight_text:}
  { \lseries }
  { \fontseries{l} \selectfont }
  {}{}
\ExplSyntaxOff

\sisetup{
    range-units=single,
    range-phrase = --,
    detect-all,
    mode = text
}

\DeclareSIUnit\tl{TL}
\DeclareSIUnit\el{EL}

\usepackage{makeidx}
\usepackage{hyperref}
\usepackage[german]{babel}

\usepackage{titlesec}
\usepackage{blindtext}
\usepackage{lipsum}
\usepackage{color}
\usepackage{xparse}

\usepackage{array}
\usepackage{tabularx}
\usepackage{booktabs}

\usepackage{titlesec}

\ifpdf
  \usepackage{pdfcolmk}
\fi

\usepackage{graphicx}

%% drawing package
\usepackage{tikz}
%% for dingbats
\usepackage{pifont}\providecommand{\HUGE}{\Huge} % if not using memoir
\newlength{\drop} % for my convenience

% Enable this package for visual debugging of boxes. Compile with LuaLaTeX.
% \usepackage{lua-visual-debug}

%% specify the Webomints family
%\newcommand*{\wb}[1]{\fontsize{#1}{#2}\usefont{U}{webo}{xl}{n}}

%% select a (FontSite) font by its font family ID
\newcommand*{\FSfont}[1]{\fontencoding{T1}\fontfamily{#1}\selectfont}

%% if you don’t have the FontSite fonts either
\renewcommand*{\FSfont}[1]{}

%% or use your own choice of family.
%% select a (TeX Font) font by its font family ID
\newcommand*{\TXfont}[1]{\fontencoding{T1}\fontfamily{#1}\selectfont}

%% Some shades
\definecolor{Dark}{gray}{0.2}
\definecolor{MedDark}{gray}{0.4}
\definecolor{Medium}{gray}{0.6}
\definecolor{Light}{gray}{0.8}

%%%% Additional font series macros
\makeatletter
%%%% light series
%% e.g., kernel doc, section s: line 12 or thereabouts
\DeclareRobustCommand\ltseries{\not@math@alphabet\ltseries\relax\fontseries\ltdefault\selectfont}
%% e.g., kernel doc, section t: line 32 or thereabouts
\newcommand{\ltdefault}{l}
%% e.g., kernel doc, section v: line 19 or thereabouts
\DeclareTextFontCommand{\textlt}{\ltseries}

% heavy(bold) series
\DeclareRobustCommand\hbseries{\not@math@alphabet\hbseries\relax\fontseries\hbdefault\selectfont}
\newcommand{\hbdefault}{hb}\DeclareTextFontCommand{\texthb}{\hbseries}
\makeatother

\setlength{\parindent}{0pt}

\def\arraystretch{1}%\tabcolsep=14pt

% fixed width column types with alignment
\newcolumntype{L}[1]{>{\raggedright\let\newline\\\arraybackslash\hspace{0pt}}m{#1}}
\newcolumntype{C}[1]{>{\centering\let\newline\\\arraybackslash\hspace{0pt}}m{#1}}
\newcolumntype{R}[1]{>{\raggedleft\let\newline\\\arraybackslash\hspace{0pt}}m{#1}}

% some colors
\definecolor{gray75}{gray}{0.75}

% chapter header style
\newcommand{\hsp}{\hspace{20pt}}
\titleformat{\chapter}[hang]{\Huge\bfseries}{\thechapter\hsp\textcolor{gray75}{|}\hsp}{0pt}{\Huge\bfseries}

\renewcommand{\footnoterule}{%
  \kern-3pt
  \hrule width \textwidth height 0.4pt
  \kern2.6pt
}

\titlespacing*{\chapter}
{0pt}{4ex plus 1ex minus 0.5ex}{4ex plus 0.5ex}
\titlespacing*{\section}
{0pt}{4ex}{3ex}
\titlespacing*{\subsection}
{0pt}{2ex}{1ex}

% Some spacing between paragraphs
\setlength{\parskip}{1em}
% Define environments and commands for recipe

\newenvironment{ingredients}[1]
{
  \vspace{0.3cm}
  \begin{tabular}{ R{1cm}@{\,}L{1cm} L{10cm}}
  & & \textbf{\large #1}\vspace{0.2cm}  \\
}
{
\end{tabular}
}

\newenvironment{recipestep}[1]
{
  \subsection*{#1}
}
{
}

\NewDocumentEnvironment{recipeintro}{ooo}
{
  \begin{tabular}[t]{@{}p{11cm}@{}p{3cm}@{}}
}
{
  &  \begin{tabular}[t]{@{}R{1cm}@{ }L{1.8cm}@{}}
       \IfNoValueTF{#1}{}{#1 \\}
       \IfNoValueTF{#2}{}{#2 \\}
     \end{tabular} \\
     \\
  \IfNoValueTF{#3}{}{Quelle: #3 \\}
\end{tabular}\par
\vspace{0.4cm}
}

\makeatletter
\newcommand \indexdotfill {\leavevmode \cleaders \hb@xt@ .44em{\hss .\hss }\hfill \kern \z@}
\makeatother

\begin{filecontents*}{\jobname.ist}
  headings_flag 1
  heading_prefix "{\\textbf{"
  heading_suffix "}}\\nopagebreak\n"
  delim_0 " \\dotfill "
  delim_1 " \\dotfill "
  delim_2 " \\dotfill "
\end{filecontents*}

\newcommand{\printindices}
{
  \printindex
  \printindex[region]
  \printindex[ingredient]
}


\makeindex[title=Index --- Kategorie, intoc, options=-s \jobname]
\makeindex[name=region,title=Index --- Region, intoc, options=-s \jobname]
\makeindex[name=ingredient,title=Index --- Zutat, intoc, options=-s \jobname]

\begin{document}

\pagenumbering{gobble}

% Titelseite
\newcommand*{\titleDB}{\begingroup% AW, Design of Books
\FSfont{5pl} % FontSite URW Palladio (Palatino)
\drop = 0.14
\textheight
\centering
\vspace*{\drop}
{\Large TURBO}\\
[\baselineskip]{\Huge IN DIE FRESSE}\\
[\baselineskip]{\Huge GEILES}\\
[\baselineskip]{\Huge KOCHBUCH}\\
[5\baselineskip]
\baselineskip25pt{\LARGE MONA SCHÖNLEBEN-FREUDENBERG \\ MARC FREUDENBERG \\ DIE GÖNNHARDTS \\ et.\ al.}\par
{mit Allem ohne Fisch\footnote{mit Fisch}}
\vfill
{\small\sffamily \today}\par\vspace{0.5cm}\endgroup}

\titleDB


% Mindestens eine leere linke Seite nach Titelseite
\cleardoublepage{}

% Inhaltsverzeichnis
\tableofcontents

% Neue Seite nach Inhaltsverzeichnis
\clearpage

% Verhindert die doppelt leere Seite nach dem Inhaltsverzeichnis.
% TODO: Evtl. doch drin lassen, damit das erste Kapitel auf einer rechten Seite startet.
\let\cleardoublepage=\clearpage

% Seitenzahlen: arabisch
\pagenumbering{arabic}

% Kapitel: Danksagungen
\chapter*{Danksagungen}
\addcontentsline{toc}{chapter}{Danksagungen}


% Kapitel: Einleitung
\chapter*{Einleitung}
\addcontentsline{toc}{chapter}{Einleitung}


% Kapitel: Technik/Ausstattung
\chapter{Technik \& Ausstattung}

% Kapitel: Basisrezepte
\chapter{Basisrezepte}\label{chp:basisrezepte}
Bitte benutzt eine Waage.

% Überschrift/Name des Rezepts
\section{Brandmasse}\label{rcp:brandmasse}%
\index{Teig}\index[region]{Deutschland}
% Der Text im 'label'-Feld erlaubt es im Dokument auf dieses Rezept zu verweisen.
% Sehr praktisch, wenn z.B. ein anderes Rezept dieses hier verwendet. Dabei werden Seitenzahl etc. automatisch
% angepasst.

% Das Folgende ist ein Einführungstext mit etwas Zusatzinfos:
\begin{recipeintro}
  % Anzahl der Portionen (optional)
  []
  % Zubereitsungsdauer (optional)
  []
  % Optional: Quellenangabe
  [Martin Schönleben]
  % Einführungstext
  Dieses Grundrezept für eine Brandmasse (Brandteig) kann als Basis für
  verschiedenste Sachen dienen, z.B. Käse-Fours, Windbeutel oder Profiterol.
\end{recipeintro}

% Ein Zutatenblock mit Überschrift
% Bei längeren/komplizierteren Rezepten bietet es sich an, mehrere solcher Blöcke zu verwenden
% um die Zutanten semantisch nach Rezeptmodule aufzuteilen.
\begin{ingredients}
  % Überschrift
  {Zutaten}
  % Zutanten sind in einer Tabelle aufgelistet. Die Spalten werden durch ein '&' Zeichen getrennt,
  % Eine Zeile wird mit '\\' abgeschlossen.

  % <Menge>     &     <Einheit>          &  <Name>     \\
  350  &  \si{\milli\litre}  &  Milch  \\
  50   &  \si{\gram}         &  Butter  \\
  250  &  \si{\gram}         &  Mehl  \\
       &                     &  geriebene Zitrone  \\
  250  &  \si{\milli\litre}  &  Vollei  \\

  % Der Befehl '\si{<einheit>}' sollte wenn möglich für Einheiten verwendet werden.
  % Wenn Einheiten fehlen, kann ich sie hinzufügen. FÜr mehr Infos: Dokumentation Paket siunitx.
  % Beispiele:

  % Teelöffel:   \si{\tl}
  % Esslöffel:   \si{\el}
  % Milliliter:  \si{\milli\litre}
  % Kilogram:    \si{\kilo\gram}
\end{ingredients}

% Ein bissl Luft zwischen Rezeptliste und Anweisungen.
\vspace{0.5cm}

% Ein Zubereitungsschritt mit Überschrift
% Diese Blöcke können beliebig oft nach einander erscheinen
\begin{recipestep}
  % Überschrift
  {Zubereitung}
  % Der Text, der die Zubereitung beschreibt:
  Butter und Milch kochen. Dann das Mehl kurz unterrühren. So lange rühren, bis sich
  die Masse ballt und einen kompakten Kloß bildet, dann vom Herd nehmen. Anschließend
  die Eier nach und nach unterrühren. Mit einer Lochtülle kleine Tupfen aufdressieren.

  Backen: \SI{190}{\degree\celsius} ca. \SIrange{15}{20}{\minute}.\\
  Beim Backen Dampf geben.
\end{recipestep}



% Kapitel: Herzhaft
\chapter{Herzhaft}

\section{Szechuan Dan Dan Nudeln}\label{rcp:szechuan-dan-dan-nudeln}%
\index{Hauptgericht}\index[region]{China}\index[ingredient]{Chilli}\index[ingredient]{Nudeln}
\begin{recipeintro}
  [2 & Portionen]
  [\nicefrac{1}{2} & Stunde]
  [\url{https://www.marionskitchen.com}]
  Turbo scharf!
\end{recipeintro}

\begin{ingredients}{Scharfe Sauce}
  2                &  \si{\tl}            &  Szechuanpfeffer, ganz  \\
  60               &  \si{\milli\litre}   &  Sojasauce              \\
  3                &  \si{\el}            &  \nameref{rcp:szechuan-chillioel}, S. \pageref{rcp:szechuan-chillioel}  \\
  1                &  \si{\el}            &  Schwarzer chinesischer Essig  \\
  2                &  \si{\tl}            &  Süße dunkle Sojasauce  \\
  \nicefrac{1}{4}  &  \si{\tl}            &  Zucker  \\
\end{ingredients}

\begin{ingredients}{Stir-Fry}
  2      &  \si{\el}            &  Pflanzenöl (z.B. Erdnussöl)  \\
  200    &  \si{\gram}          &  Schweinehack \\
  2      &  \si{\el}            &  Tianjin eingelegte Senfblätter (Ya Chai) oder Kohl (Tong Chai) \\
  1      &  \si{\el}            &  Sojasauce  \\
  1      &  \si{\el}            &  Shaoxing (Chinesischer Kochwein)  \\
\end{ingredients}

\begin{ingredients}{Anrichten}
  200    &  \si{\gram}         &  Dünne, chinesische Weizennudeln  \\
         &                 &  Frühlingszwiebeln zum Anrichten  \\
\end{ingredients}

\vspace{0.5cm}

\begin{recipestep}{Scharfe Sauce}
  Den Szechuanpfeffer in einer Pfanne \SIrange{2}{3}{\minute} ohne Öl bei mittlerer Hitze rösten bis er beginnt zu duften.
  In einem Mörser fein zerstoßen und beiseite stellen. Alle anderen Zutaten in einer großen Schüssel verühren und beiseite stellen.
\end{recipestep}

\begin{recipestep}{Stir-Fry}
  Das Öl in einem Wok oder einer großen Pfanne auf hoher Hitze stark erhitzen. Das Hackfleisch hineingeben und verteilen.
  Bei größeren Mengen unbedingt nicht zu viel Fleisch auf einmal verwenden! Es muss stark angebraten werden und nicht dünsten!
  Unter gelegentlichem Rühren kräftig anbraten, bis es eine schöne gold-braune Farbe hat. Nun stärker rühren und das Fleisch fein
  auftrennen, schließlich den gerösteten und zerstoßenen Szechuanpfeffer einarbeiten. Die restlichen Zutaten dazugeben und unter
  ständigem Rühren weiterbraten so dass das Fleisch die Flüssigkeit aufnehmen kann. Beiseite stellen und warm halten, während die Nudeln kochen.

  Einen großen Topf Wasser zum Kochen bringen und die Nudeln nach Packungsanweisungen kochen. Während die Nudeln kochen, einen kleinen Schöpfer
  Nudelwasser (etwa \SIrange{50}{100}{\milli\litre}) entnehmen und mit der Soße vermischen. Die Frühlingszwiebeln in feine Ringe schneiden.
\end{recipestep}

\begin{recipestep}{Anrichten}
  Etwas scharfe Soße in tiefe Teller oder Schüsseln geben, die Nudeln darin zu einem Nest drehen.
  Die Fleischmischung darauf anrichten und mit den Frühlingszwiebeln garnieren. Übrige scharfe Soße und Chilliöl
  zum Nachwürzen anbieten.
\end{recipestep}

\section{Szechuan Chilliöl}\label{rcp:szechuan-chillioel}%
\index{Zutat}\index[region]{China}\index[ingredient]{Chilli}\index[ingredient]{Oel@"Öl}
\begin{recipeintro}
  [2 & Portionen]
  [\nicefrac{1}{4} & Stunde]
  [\url{https://www.marionskitchen.com}]
  Universelles, chinesisches Chilliöl.
\end{recipeintro}

\begin{ingredients}{Zutaten}
  200  &  \si{\milli\litre}  &  Erdnussöl  \\
  4    &  Stk.               &  Zimtstange  \\
  2    &  Stk.               &  Lorbeerblätter  \\
  3    &  \si{\el}           &  Szechuan-Pfefferkörner  \\
  8    &  Stk.               &  Kardamomkapseln  \\
  80   &  \si{\milli\litre}  &  Asiatische Chilliflocken  \\
  2    &  \si{\tl}           &  Salz  \\
\end{ingredients}

\vspace{0.5cm}

\begin{recipestep}{Zubereitung}
  In einer Pfanne das Erdnussöl bei geringer Hitze erwärmen.\par

  Die Gewürze in einer weiteren Pfanne ohne Öl bei mittlerer Hitze rösten, gerade bis sie beginnen leicht zu rauchen.
  In die Pfanne mit dem warmen Öl geben, unter mittlerer Hitze erhitzen und etwa \SIrange{4}{5}{\minute} sanft braten.
  Beiseite stellen und abkühlen lassen.\par

  Chilliflocken und Salz in eine hitzefeste Schüssel oder Glas geben (z.B. Einmachglas) und das Öl hineinseien.
  Lorbeerblätter, Zimtstange und Sternanis ebenfalls hinzufügen.\par
\end{recipestep}

\section{Blanchierter Pak Choi}\label{rcp:blanchierter-pak-choi}
\begin{recipeintro}
  [2 & Portionen]
  [\nicefrac{1}{4} & Stunde]
  []
  [\index{Beilage}\index[region]{China}\index[ingredient]{Pak Choi}]
  Eine klassische chinesische Beilage.
\end{recipeintro}

\begin{ingredients}{Zutaten}
  1  &  Stk.      &  Pak Choi  \\
  1  &  \si{\el}  &  Austernsauce  \\
  1  &  \si{\tl}  &  Zucker  \\
  1  &  \si{\el}  &  Erdnussöl \\
  1  &  \si{\el}  &  Wasser \\
\end{ingredients}

\vspace{0.5cm}

\begin{recipestep}{Zubereitung}
  Einen Topf mit reichlich gesalzenem Wasser zum Kochen bringen. In der Zwischenzeit die Austernsauce, den Zucker, das Wasser und das Erdnussöl
  in einem Topf oder Wok unter Rühren erwärmen bis sich der Zucker vollständig aufgelöst hat.\par

  Die Blätter des Pak Choi abziehen, waschen und je nach Größe vierteln oder halbieren. Eine Minute blanchieren, herausnehmen und dann sofort
  mit der lauwarmen Sauce vermischen.
\end{recipestep}


\section{Grießnockerl --- Mona}\label{rcp:griessnockerl-mona}

\begin{recipeintro}
  [4 & Portionen]
  [\nicefrac{1}{2} & Stunde]
  [Beate Schönleben]
  [\index{Beilage}\index{Zutat}\index[region]{Deutschland}]
  Klassische, feine Grießnockerl nach Familienrezept. Sehr fluffig und weich, diese Nockerl zergehen auf der Zunge!
  % Für eine etwas festere Variante siehe: \nameref{rcp:griessnockerl-marc}, S.~\pageref{rcp:griessnockerl-marc}.
\end{recipeintro}

\begin{ingredients}{Zutaten}
  125  &  \si{\milli\litre}  &  Milch                     \\ % TODO: Diese Menge überprüfen, im Rezept steht "1 Tasse"
  2    &  \si{\el}           &  Butter                    \\ % TODO: Bitte noch in Gramm angeben
  2    &  Stk.               &  Eier                      \\
  4    &  \si{\el}           &  Hartweizengrieß           \\
       &                     &  Salz                      \\ % TODO: Mengenangabe fehlt!
  1    &  Prise              &  Muskatnuss, gerieben      \\
       &                     &  Petersilie, fein gehackt  \\ % TODO: Mengenangabe fehlt!
\end{ingredients}

\vspace{0.5cm}

\begin{recipestep}{Zubereitung}
  Milch, Butter, Salz und Muskat zum Kochen bringen.\par
  Grieß einrühren, Hitze reduzieren.\par
  Eier verkleppern und einrühren.\par  % TODO: Das bitte verifizieren! Habe das so angenommen, ist im Quellrezept leider nicht angegeben.
  Mit befeuchteten Löffeln zu Nockerl abstechen.\par % TODO: Referenz auf Technikerlkärung?
  In kochende Brühe oder kräftig gesalzenes Wasser einlegen. Etwa 5 Minuten kochen
  und anschließend weitere 10 bis 15 Minuten ziehen lassen, mindestens bis die Nockerl
  aufschwimmen.
\end{recipestep}


%   \section{Grießnockerl --- Marc}\label{rcp:griessnockerl-marc}
%
%   \begin{recipeintro}[4 & Portionen][\nicefrac{1}{2} & Stunde][Eva-Maria Freudenberg]
%     Diese etwas festere Grießnockerlvariante ist Marcs Favorit.
%   \end{recipeintro}
%
%   % TODO: REZEPT EINHOLEN!
%   \begin{ingredients}{Zutaten}
%   125  &  \si{\milli\litre}  &  Milch                     \\ % TODO: Diese Menge überprüfen, im Rezept steht "1 Tasse"
%   2    &  \si{\el}           &  Butter                    \\ % TODO: Bitte noch in Gramm angeben
%   2    &  Stk.               &  Eier                      \\
%   4    &  \si{\el}           &  Hartweizengrieß           \\
%        &                     &  Salz                      \\ % TODO: Mengenangabe fehlt!
%   1    &  Prise              &  Muskatnuss, gerieben      \\
%        &                     &  Petersilie, fein gehackt  \\ % TODO: Mengenangabe fehlt!
%   \end{ingredients}
%
%   \vspace{0.5cm}
%
%   \begin{recipestep}{Zubereitung}
%
%   \end{recipestep}

\section{Schwäbische Spätzle}\label{rcp:schwaebische-spaetzle}%
\index{Beilage}\index{Zutat}\index[region]{Deutschland}
\begin{recipeintro}
  []
  []
  [Irgendein Kochbuch-klassiker] % TODO: Genauer herausfinden!
  Knöpfleförmige Spätzle!
\end{recipeintro}

\begin{ingredients}
  {Zutaten}
  500  &  \si{\gram}         &  Weizenmehl  \\
  4    &  Stk.               &  Eier        \\
  1    &  \si{\tl}           &  Salz        \\  % TODO: Ich glaube, das könnte mehr sein!
  250  &  \si{\milli\litre}  &  Wasser      \\
\end{ingredients}

\vspace{0.5cm}

\begin{recipestep}
  {Zubereitung}
  Mehl in eine Schüssel geben, in die Mitte eine Vertiefung machen, die Eier
  hineinschlagen. Unter Zugießen des kalten Wassers den Teig mit einem
  Löffel rühren und schlagen, bis er Blasen wirft.
  Den Spätzlesteig in \SIrange{2}{3}{\litre} kochendes Wasser entweder in dünnen
  Streifen mit einem Spätzleschaber bzw. mit einem größeren Messer vom
  angefeuchteten Spätzlesbrett schnell hineinschaben, oder den Teig durch die
  Spätzlesmaschine drücken. Dabei immer nur kleine Portionen auf einmal ins Wasser
  geben. Spätzle kurz kochen, bis sie oben schwimmen, mit einem Schaumlöffel
  herausheben und auf einem mit Backpapier ausgelegten Blech auskühlen lassen.
\end{recipestep}


% Kapitel: Süß
\chapter{Süß}
% Überschrift/Name des Rezepts
\section{Mousse au Chocolate}\label{rcp:mousse-au-chocolate}
% Der Text im 'label'-Feld erlaubt es im Dokument auf dieses Rezept zu verweisen.
% Sehr praktisch, wenn z.B. ein anderes Rezept dieses hier verwendet. Dabei werden Seitenzahl etc. automatisch
% angepasst.

% Das Folgende ist ein Einführungstext mit etwas Zusatzinfos:
\begin{recipeintro}
  % Anzahl der Portionen (optional)
  []
  % Zubereitsungsdauer (optional)
  []
  % Optional: Quellenangabe
  []

  % Einführungstext
  Super geiles Rezept!
\end{recipeintro}

% Ein Zutatenblock mit Überschrift
% Bei längeren/komplizierteren Rezepten bietet es sich an, mehrere solcher Blöcke zu verwenden
% um die Zutanten semantisch nach Rezeptmodule aufzuteilen.
\begin{ingredients}
  % Überschrift
  {Zutaten}
  % Zutanten sind in einer Tabelle aufgelistet. Die Spalten werden durch ein '&' Zeichen getrennt,
  % Eine Zeile wird mit '\\' abgeschlossen.

  % <Menge>     &     <Einheit>          &  <Name>     \\
  2                &  \si{\tl}           &  Szechuanpfeffer, ganz  \\

  % Der Befehl '\si{<einheit>}' sollte wenn möglich für Einheiten verwendet werden.
  % Wenn Einheiten fehlen, kann ich sie hinzufügen. FÜr mehr Infos: Dokumentation Paket siunitx.
  % Beispiele:

  % Teelöffel:   \si{\tl}
  % Esslöffel:   \si{\el}
  % Milliliter:  \si{\milli\litre}
  % Kilogram:    \si{\kilo\gram}
\end{ingredients}

% Ein bissl Luft zwischen Rezeptliste und Anweisungen.
\vspace{0.5cm}

% Ein Zubereitungsschritt mit Überschrift
% Diese Blöcke können beliebig oft nach einander erscheinen
\begin{recipestep}
  % Überschrift
  {Zubereitung}
  % Der Text, der die Zubereitung beschreibt:
  Bla bla bla\ldots
\end{recipestep}


% Kapitel: Sonstiges
\chapter{Sonstiges}

% Kapitel: Index
\printindices%

\end{document}