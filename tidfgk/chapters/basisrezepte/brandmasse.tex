% Überschrift/Name des Rezepts
\section{Brandmasse}\label{rcp:brandmasse}%
\index{Teig}\index[region]{Deutschland}
% Der Text im 'label'-Feld erlaubt es im Dokument auf dieses Rezept zu verweisen.
% Sehr praktisch, wenn z.B. ein anderes Rezept dieses hier verwendet. Dabei werden Seitenzahl etc. automatisch
% angepasst.

% Das Folgende ist ein Einführungstext mit etwas Zusatzinfos:
\begin{recipeintro}
  % Anzahl der Portionen (optional)
  []
  % Zubereitsungsdauer (optional)
  []
  % Optional: Quellenangabe
  [Martin Schönleben]
  % Einführungstext
  Dieses Grundrezept für eine Brandmasse (Brandteig) kann als Basis für
  verschiedenste Sachen dienen, z.B. Käse-Fours, Windbeutel oder Profiterol.
\end{recipeintro}

% Ein Zutatenblock mit Überschrift
% Bei längeren/komplizierteren Rezepten bietet es sich an, mehrere solcher Blöcke zu verwenden
% um die Zutanten semantisch nach Rezeptmodule aufzuteilen.
\begin{ingredients}
  % Überschrift
  {Zutaten}
  % Zutanten sind in einer Tabelle aufgelistet. Die Spalten werden durch ein '&' Zeichen getrennt,
  % Eine Zeile wird mit '\\' abgeschlossen.

  % <Menge>     &     <Einheit>          &  <Name>     \\
  350  &  \si{\milli\litre}  &  Milch  \\
  50   &  \si{\gram}         &  Butter  \\
  250  &  \si{\gram}         &  Mehl  \\
       &                     &  geriebene Zitrone  \\
  250  &  \si{\milli\litre}  &  Vollei  \\

  % Der Befehl '\si{<einheit>}' sollte wenn möglich für Einheiten verwendet werden.
  % Wenn Einheiten fehlen, kann ich sie hinzufügen. FÜr mehr Infos: Dokumentation Paket siunitx.
  % Beispiele:

  % Teelöffel:   \si{\tl}
  % Esslöffel:   \si{\el}
  % Milliliter:  \si{\milli\litre}
  % Kilogram:    \si{\kilo\gram}
\end{ingredients}

% Ein bissl Luft zwischen Rezeptliste und Anweisungen.
\vspace{0.5cm}

% Ein Zubereitungsschritt mit Überschrift
% Diese Blöcke können beliebig oft nach einander erscheinen
\begin{recipestep}
  % Überschrift
  {Zubereitung}
  % Der Text, der die Zubereitung beschreibt:
  Butter und Milch kochen. Dann das Mehl kurz unterrühren. So lange rühren, bis sich
  die Masse ballt und einen kompakten Kloß bildet, dann vom Herd nehmen. Anschließend
  die Eier nach und nach unterrühren. Mit einer Lochtülle kleine Tupfen aufdressieren.

  Backen: \SI{190}{\degree\celsius} ca. \SIrange{15}{20}{\minute}.\\
  Beim Backen Dampf geben.
\end{recipestep}
