\section{Schwäbische Spätzle}\label{rcp:schwaebische-spaetzle}%
\index{Beilage}\index{Zutat}\index[region]{Deutschland}
\begin{recipeintro}
  []
  []
  [Irgendein Kochbuch-klassiker] % TODO: Genauer herausfinden!
  Knöpfleförmige Spätzle!
\end{recipeintro}

\begin{ingredients}
  {Zutaten}
  500  &  \si{\gram}         &  Weizenmehl  \\
  4    &  Stk.               &  Eier        \\
  1    &  \si{\tl}           &  Salz        \\  % TODO: Ich glaube, das könnte mehr sein!
  250  &  \si{\milli\litre}  &  Wasser      \\
\end{ingredients}

\vspace{0.5cm}

\begin{recipestep}
  {Zubereitung}
  Mehl in eine Schüssel geben, in die Mitte eine Vertiefung machen, die Eier
  hineinschlagen. Unter Zugießen des kalten Wassers den Teig mit einem
  Löffel rühren und schlagen, bis er Blasen wirft.
  Den Spätzlesteig in \SIrange{2}{3}{\litre} kochendes Wasser entweder in dünnen
  Streifen mit einem Spätzleschaber bzw.\ mit einem größeren Messer vom
  angefeuchteten Spätzlesbrett schnell hineinschaben, oder den Teig durch die
  Spätzlesmaschine drücken. Dabei immer nur kleine Portionen auf einmal ins Wasser
  geben. Spätzle kurz kochen, bis sie oben schwimmen, mit einem Schaumlöffel
  herausheben und auf einem mit Backpapier ausgelegten Blech auskühlen lassen.
\end{recipestep}
