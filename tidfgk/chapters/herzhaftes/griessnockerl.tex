\section{Grießnockerl --- Mona}\label{rcp:griessnockerl-mona}%
\index{Beilage}\index{Zutat}\index[region]{Deutschland}
\begin{recipeintro}
  [4 & Portionen]
  [\nicefrac{1}{2} & Stunde]
  [Beate Schönleben]
  Klassische, feine Grießnockerl nach Familienrezept. Sehr fluffig und weich, diese Nockerl zergehen auf der Zunge!
  % Für eine etwas festere Variante siehe: \nameref{rcp:griessnockerl-marc}, S.~\pageref{rcp:griessnockerl-marc}.
\end{recipeintro}

\begin{ingredients}{Zutaten}
  125  &  \si{\milli\litre}  &  Milch                     \\ % TODO: Diese Menge überprüfen, im Rezept steht "1 Tasse"
  2    &  \si{\el}           &  Butter                    \\ % TODO: Bitte noch in Gramm angeben
  2    &  Stk.               &  Eier                      \\
  4    &  \si{\el}           &  Hartweizengrieß           \\
       &                     &  Salz                      \\ % TODO: Mengenangabe fehlt!
  1    &  Prise              &  Muskatnuss, gerieben      \\
       &                     &  Petersilie, fein gehackt  \\ % TODO: Mengenangabe fehlt!
\end{ingredients}

\vspace{0.5cm}

\begin{recipestep}{Zubereitung}
  Milch, Butter, Salz und Muskat zum Kochen bringen.\par
  Grieß einrühren, Hitze reduzieren.\par
  Eier verkleppern und einrühren.\par  % TODO: Das bitte verifizieren! Habe das so angenommen, ist im Quellrezept leider nicht angegeben.
  Mit befeuchteten Löffeln zu Nockerl abstechen.\par % TODO: Referenz auf Technikerlkärung?
  In kochende Brühe oder kräftig gesalzenes Wasser einlegen. Etwa 5 Minuten kochen
  und anschließend weitere 10 bis 15 Minuten ziehen lassen, mindestens bis die Nockerl
  aufschwimmen.
\end{recipestep}


%   \section{Grießnockerl --- Marc}\label{rcp:griessnockerl-marc}
%
%   \begin{recipeintro}[4 & Portionen][\nicefrac{1}{2} & Stunde][Eva-Maria Freudenberg]
%     Diese etwas festere Grießnockerlvariante ist Marcs Favorit.
%   \end{recipeintro}
%
%   % TODO: REZEPT EINHOLEN!
%   \begin{ingredients}{Zutaten}
%   125  &  \si{\milli\litre}  &  Milch                     \\ % TODO: Diese Menge überprüfen, im Rezept steht "1 Tasse"
%   2    &  \si{\el}           &  Butter                    \\ % TODO: Bitte noch in Gramm angeben
%   2    &  Stk.               &  Eier                      \\
%   4    &  \si{\el}           &  Hartweizengrieß           \\
%        &                     &  Salz                      \\ % TODO: Mengenangabe fehlt!
%   1    &  Prise              &  Muskatnuss, gerieben      \\
%        &                     &  Petersilie, fein gehackt  \\ % TODO: Mengenangabe fehlt!
%   \end{ingredients}
%
%   \vspace{0.5cm}
%
%   \begin{recipestep}{Zubereitung}
%
%   \end{recipestep}
